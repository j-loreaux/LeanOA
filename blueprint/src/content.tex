% In this file you should put the actual content of the blueprint.
% It will be used both by the web and the print version.
% It should *not* include the \begin{document}
%
% If you want to split the blueprint content into several files then
% the current file can be a simple sequence of \input. Otherwise It
% can start with a \section or \chapter for instance.

\section{WStarAlgebras and their Topologies}

In this section we define the topologies and need to prove that $W^{*}$--algebras are closed with respect to these. Must also put these
dependencies explicitly into the results that use them below. 

Maybe collect things into Basic CStar section and Basic WStar section?

\section{Order Lemmas}

Let $\mathcal{A}$ be a unital $C^{*}$--algebra in this section. We collect lemmas for the ordering of elements in $\mathcal{A}$. 
Recall that a selfadjoint element $a \in \mathcal{A}$ is said to be \textbf{positive} if its spectrum is contained in $\mathbb{R}_{\ge 0}$. 
This is written $a\ge 0$. If $a,b \in \mathcal{A}$ are selfadjoint then $b \le a$ if $b-a \ge 0$.
(TO DO: adapt this to Jireh's fully general ordering on all elements without mentioning the selfadjoint ones.)

\begin{lemma}
  \label{lem:pos_iff_star_mul_self_Sak_1_4_4} (Sak 1.4.4)
  Let $h \in \mathcal{A}$. TFAE:
  \begin{enumerate}
  \item $h \ge 0$;
  \item There exists $x\in \mathcal{A}$ such that $h=x^{*}x$.
  \end{enumerate}
\end{lemma}

\begin{proof}
\end{proof}

\begin{corollary}
  \label{lem:star_conj_pos}
  \uses{lem:pos_iff_star_mul_self_Sak_1_4_4}
  If $h,a \in \mathcal{A}$ with $h\ge 0$ then $a^{*}ha\ge 0$
\end{corollary}

\begin{proof}
  By Lemma \ref{lem:pos_iff_star_mul_self_Sak_1_4_4}, $a^{*}ha=a^{*}x^{*}xa=(xa)^{*}(xa)\ge 0$.
\end{proof}

\begin{lemma}
  \label{lem:selfadjoint_le_norm}
  \lean{IsSelfAdjoint.le_algebraMap_norm_self}
  \leanok
  \mathlibok
  If $x\in \mathcal{A}$ is selfadjoint then $\|x\|1-x \ge 0$.
\end{lemma}

\begin{proof}
\leanok
\end{proof}

\section{Projection Lemmas}

Let $\mathcal{A}$ be a unital $C^{*}$--algebra in this section. In this section we collect relevant results about (selfadjoint) projections in $\mathcal{A}$. 
Recall that an element $p$ in a $C^{*}$--algebra is a projection if $p^2=p^{*}=p$. By the Spectral Mapping Theorem (or using the full Gelfand Duality),
a selfadjoint element $a\in A$ is a projection if and only if the spectrum of $a$ is contained in $\{0,1\}$.

\begin{lemma}
  \label{lem:proj_le_one}
  \uses{lem:selfadjoint_le_norm}
  For all projections $p\in \mathcal{A}$, p \le 1.
\end{lemma}

\begin{proof}
  We can get this by Sakai 1.2.3. and Lemma \ref{lem:selfadjoint_le_norm} We should include that result here.
\end{proof}

\begin{lemma}
  \label{lem:proj_sub_pos_iff_comm_eq_self}
  \uses{lem:star_conj_pos, lem:proj_le_one}
  Let $\mathcal{A}$ be a $C^{*}$--algebra and $p,q \in \mathcal{A}$ be projections. Then $p-q\ge 0$ iff $qp = pq = q$.
\end{lemma}

\begin{proof}
  By Lemma \ref{lem:star_conj_pos}, if $p-q \ge 0$ then $q(p-q)q=qpq-q^3=qpq-q \ge 0$. By Lemma \ref{lem:proj_le_one}, we have $p\le 1$ and 
  employing Lemma \ref{lem:star_conj_pos} we obtain $qpq\le q1q=q$. Since $qpq\ge q$ and $qpq\le q$ we have $qpq=q$, which implies that $q(p-q)q=0$.
  By the $C^{*}$-property of the norm, we have $\|(p-q)^{1/2}q\|^2=\|q(p-q)q\|=0$ and so $(p-q)^{1/2}q=0$ and therefore $(p-q)q=(p-q)^{1/2}(p-q)^{1/2}q=0$.
  It follows that $pq=q$, and taking adjoints, $qp=q$.
  Conversely, if $qp=pq=q$, one easily checks that $p-q$ is a projection and so its spectrum is contained in $\{0,1\}$ and it is positive.
\end{proof}

\begin{corollary}
  \label{cor:proj_sub_of_subproj}
  \uses{lem:proj_sub_pos_iff_comm_eq_self}
  For all $p,q \in \mathcal{A}$ projections such that $q\le p$, $p-q$ is a projection.
\end{corollary}

\section{Positive Linear Functionals}

\section{Normality and Ultraweak Continuity for Positive Functionals}

In what follows, let $M$ be a (nonzero) $W^{*}$--algebra. Let $\mathcal{P}(M)$
denote the projection lattice of $M$. Let $\varphi$ be a positive linear functional $\varphi$ on $M$.
We say $\varphi$ is \textbf{normal} if whenever $(p_{\alpha})$ is an increasing net of projections in $M$ with supremum $p$,
we have $\varphi(p_{\alpha})\to \varphi(p)$. In this section we show that this property is equivalent to 
$\sigma(M, M_{*})$--continuity. We say that a linear functional $\varphi$ on $M$
is \textbf{positive} if $\varphi(x)\ge 0$ whenever $x\ge 0$.

\subsection{Ultraweakly Continuous Implies Normal} 

This is 1.7.4 in Sakai. Roughly, taking an increasing net of projections, it converges to the supremum strongly, hence $\sigma$-topology,
and the $\sigma$-continuity finishes the proof.

\begin{theorem}
  \label{lem:sigma_cont_of_normal_Sak_1_7_4}
  Every uniformly bounded, increasing directed set converges
  to its least upper bound in the $\sigma$--topology. Further, if
  $x=\sup_{\alpha}x_\alpha$ then $a^{*}xa=\sup_{\alpha}a^{*}xa$.
\end{theorem}

\begin{proof}
\end{proof}

\subsection{Normal Implies Ultraweakly Continuous}

\begin{lemma}
  \label{lem:proj_of_sup_proj}
  For every increasing bounded net $(p_{\alpha})$ of projections in $M$, the supremum $p$
  is a projection in $M$
\end{lemma}

\begin{proof}
\end{proof}

If $P:\mathcal{P}(M)\to \text{Prop}$ is a predicate, the usual ordering ``$\le$'' on projections
induces an order on the set $\{p\in \mathcal{P}(M)|P(p)\}$. In what follows there will be no confusion if 
we also denote this induced order by ``$\le$''.

\begin{lemma}
  \label{lem:uw_pos_sep_pts}
For every nonzero element $a\in M$, there is a $\sigma(M,M_{*})$--continuous positive linear
functional $\psi$ on $M$ such that $\psi(a)\ne 0$.
\end{lemma}

\begin{proof}
\end{proof}

\begin{lemma}
  \label{lem:exists_uw_ge_normal}
  For every positive normal linear functional $\varphi$ and nonzero $p\in \mathcal{P}(M)$
  there exists a positive $\sigma(M,M_{*})$--continuous linear functional $\psi$ such that $\varphi(p)<\psi(p)$.
\end{lemma}

\begin{proof}
  By Lemma \ref{lem:uw_pos_sep_pts} there is a $\sigma(M,M_{*})$--continuous positive linear
  functional $\psi_0$ on $M$ such that $\psi_0(p)\ne 0$. Rescale this functional by a positive constant to obtain
  $\psi$ such that $\varphi(p)<\psi(p)$.
\end{proof}

\begin{lemma}
  \label{lem:msr_th_lemma}
  \uses{lem:sigma_cont_of_normal_Sak_1_7_4,lem:proj_of_sup_proj,cor:proj_sub_of_subproj}
For all positive linear functionals $\varphi,\psi$ with
$\varphi$ normal and $\psi$ $\sigma(M,M_{*})$--continuous and every nonzero $p\in \mathcal{P}(M)$
such that $\varphi(p)<\psi(p)$, there exists a nonzero $p_1 \in \mathcal{P}(M)$ such that $p_1\le p$ and for all nonzero
$q\in \mathcal{P}(M)$ with $q\le p_1$, we have $\varphi(q)<\psi(q)$. 
\end{lemma}

\begin{proof}
  We proceed by contradiction. Suppose the conclusion does not hold. Then, for every nonzero subprojection
  $p_1$ of $p$ there is a nonzero subprojection $q\le p_1$ such that $\varphi(q)\ge \psi(q)$. In particular,
  (letting $p_1=q$) there is a $q\le p$ such that $\varphi(q)\ge \psi(q)$. If $(q_\alpha)$ is a chain
  of such nonzero projections, $q_{\alpha}\to \sup q_{\alpha}$ in the $\sigma$--topology by Theorem \ref{lem:sigma_cont_of_normal_Sak_1_7_4}, 
  and by Lemma \ref{lem:proj_of_sup_proj} we know this supremum is a subprojection of $p_1$. Since $\varphi$
  is positive and normal, and $\psi$ is positive and $\sigma$--continuous, we have $\varphi(\sup q_{\alpha})\ge \psi(\sup q_{\alpha})$.
  Therefore, by Zorn's Lemma, there is a maximal $q_{0}\le p$ such that $\varphi(q_0)\ge \psi(q_0)$. We claim $q_{0}=p$. If not, 
  $p-q_0$ is a nonzero subprojection of $p$ by Corollary \ref{cor:proj_sub_of_subproj}, and there exists a nonzero projection $q_{1}\le p-q_{0}$
  such that $\varphi(q_{1})\ge \psi(q_{1})$. But then $q_{0}$ is a proper subprojection of $q_0+q_1$ and by linearity and positivity 
  $\varphi(q_0+q_1)\ge \psi(q_0+q_1)$ contradicting the maximality of $q_0$. Thus $p=q_0$ and so $\varphi(p) \ge \psi(p)$, contradicting 
  the hypothesis $\varphi(p)<\psi(p)$.  
\end{proof}

Recall that a compact Hausdorff space is \textbf{Stonean} if the closure of every open set is open.

\begin{lemma}
  \label{lem:fin_lin_approx_of_stonean_Sak_1_3_1} (Sak  1.3.1)
  Let $K$ be a Stonean space. Then every element $a$ in $C(K)$ can be uniformly approximated by 
  finite linear combinations of projections in $C(K)$. If $a\ge 0$ then the coefficients of the approximating 
  linear combinations may be chosen nonnegative.
\end{lemma}

\begin{proof}
\end{proof}

\begin{lemma}
  \label{lem:cut_down_of_wstar} 
  If $p\in M$ is a projection, $pMp$ is also a $W^{*}$ algebra with identity $p$.
\end{lemma}

\begin{proof}
\end{proof}

\begin{lemma}
  \label{lem:spec_masa_wstar_stonean_Sak_1_7_5} (Sak 1.7.5)
  If $C$ is any maximal commutative $C^*$--subalgebra of the $W^{*}$--algebra $M$, its
  spectrum space (maximal ideal space) is Stonean.
\end{lemma}

\begin{proof}
\end{proof}

Note that the following likely contains the most crucial use of the normality of $\varphi$ for the later proof 
of Theorem \ref{thm: sigma_cts_of_normal}.

\begin{lemma}
  \label{lem:zorn_base}
  \uses{cor:proj_sub_of_subproj,lem:selfadjoint_le_norm,lem:sigma_cont_of_normal_Sak_1_7_4,lem:proj_of_sup_proj}
  Let $\varphi$ be a normal positive linear functional on $M$ and consider the predicate $P:\mathcal{P}(M)\to \text{Prop}$ defined, for $p\in \mathcal{P}(M)$, by 
  ``$M\ni x \mapsto \varphi(xp)$ is $\sigma(M,M_{*})$--continuous''. If $(p_{\alpha})$ is a chain of projections in $M$ such that $P(p_{\alpha})$ is true for each $\alpha$,
  then $P(\sup(p_{\alpha}))$ is true. Hence by Zorn's Lemma there is a maximal $p_0\in \mathcal{P}(M)$ such that $P(p_0)$ is true.  
\end{lemma}

\begin{proof}
  Let $x$ be on the unit sphere and let $p$ be the supremum of the $p_{\alpha}$. By Theorem \ref{lem:sigma_cont_of_normal_Sak_1_7_4} and Lemma \ref{lem:proj_of_sup_proj} 
  we know $p_{\alpha}$ converges in $\sigma$--topology to $p$, and $p$ is a projection. 
  By Corollary \ref{cor:proj_sub_of_subproj} we know $p-p_\alpha$ is a projection. By the $C^{*}$--property
  of the norm ($\|x^{*}x\|=\|x\|^{2}$), Lemma \ref{lem:selfadjoint_le_norm} and the positivity of $\varphi$ we have $\varphi(x^{*}x)\le \varphi(1)$.
  Now these facts together with Cauchy-Schwartz and the monotonicity of square roots, 
  \begin{align}
    |\varphi(x(p-p_{\alpha}))|&\le \varphi(x^{*}x)^{1/2}\varphi(p-p_{\alpha})^{1/2}\\ \nonumber
    &\le \varphi(1)^{1/2}\varphi(p-p_{\alpha})^{1/2}.
  \end{align}
  Hence, by the definition of operator norm, $\|\varphi(\cdot(p-p_{\alpha}))\|\le \varphi(1)^{1/2}\varphi(p-p_{\alpha})^{1/2}$. The right
  hand side converges to 0 with $\alpha$ due to the normality of $\varphi$ and therefore $\varphi(\cdot p_{\alpha})$ converges to $\varphi(\cdot p)$
  in norm. Since the set of $\sigma$--continuous functionals on $M$ is norm closed, it follows that $\sigma(\cdot p)$ is $\sigma$--continuous.
  We obtain a maximal $p_{0}$ by Zorn's Lemma.
\end{proof}

\begin{lemma}
  \label{lem:bdd_sigma_cts_iff_bdd_s_cts_Sak_1_8_10} (Sak 1.8.10)
  A linear functional $\rho$ on $M$ is $\sigma$ continuous on the unit sphere (hence $\sigma$--continuous) if and only if it is $s$--continuous on the
  unit sphere (hence $s$--continuous).
\end{lemma}

\begin{proof}
\end{proof}

\begin{lemma}
  \label{lem:wstar_of_masa}
  A maximal abelian $*$--subalgebra of a $W^{*}$--algebra $M$ is also a $W^{*}$--algebra.
\end{lemma}

\begin{proof}
\end{proof}

\begin{theorem}
  \label{thm:sigma_cts_of_normal}
  \uses{lem:zorn_base,lem:exists_uw_ge_normal,lem:msr_th_lemma,lem:cut_down_of_wstar,lem:spec_masa_wstar_stonean_Sak_1_7_5,lem:fin_lin_approx_of_stonean_Sak_1_3_1,lem:bdd_sigma_cts_iff_bdd_s_cts_Sak_1_8_10,lem:wstar_of_masa}
  Every positive normal linear functional $\varphi$ on $M$ is $\sigma(M,M_{*})$--continuous.
\end{theorem}

\begin{proof}
  The claim is obvious for the zero functional. Let $\varphi$ be a nonzero positive normal linear functional. By Lemma \ref{lem:zorn_base} we have
  a maximal $p_0\in \mathcal{P}(M)$ such that $M\ni x \mapsto \varphi(xp_0)$ is $\sigma(M,M_{*})$--continuous. Assume for the purposes of finding
  a contradiction that $p_0\ne 1$. By Lemma \ref{lem:exists_uw_ge_normal}
  there is a $\sigma(M,M_{*})$--continuous positive functional $\psi$ on $M$ such that $\varphi(1-p_0)<\psi(1-p_0)$. By Lemma \ref{lem:msr_th_lemma}
  there is a nonzero subprojection $p\le 1-p_0$ in $M$ such that $\varphi(q)< \psi(q)$ for every nonzero $q\le p$ in $M$. Let $x\in pMp$ be on the unit sphere. 
  Then $x^{*}x$ is positive and hence normal, so the $C^{*}$--subalgebra of $pMp$ generated by $x^{*}x$ and $p$ is commutative, and is hence contained in a maximal abelian
  $*$--subalgebra $A$ of $pMp$. Now $A$ is a $W^{*}$--subalgebra of $pMp$ by Lemma \ref{lem:wstar_of_masa} and hence is a maximal commutative $C^{*}$--subalgebra of $pMp$. Via the Gelfand Transform, 
  $A$ is star isomorphic to $C(K)$, where $K$ is Stonean by Lemmas \ref{lem:cut_down_of_wstar} and \ref{lem:spec_masa_wstar_stonean_Sak_1_7_5}. 
  By Lemma \ref{lem:fin_lin_approx_of_stonean_Sak_1_3_1} it follows that $\varphi(a)\le \psi(a)$ for every $a\ge 0$ in $A$, which holds a fortiori for $a\ge 0$ in $C^{*}(x^{*}x,p)$. 
  In particular, $\varphi(px^{*}xp)\le \psi(px^{*}xp)$. Therefore,
  \begin{align}
    |\varphi(x(p_0+p))|&\le |\varphi(xp_0)|+|\varphi(xp)|\\ \nonumber
    &\le |\varphi(xp_0)|+\varphi(1)^{1/2}\varphi(px^{*}xp)^{1/2}\\ \nonumber
    &\le |\varphi(xp_0)|+\varphi(1)^{1/2}\psi(px^{*}xp)^{1/2}.
  \end{align}
  Since $x\mapsto \varphi(xp_0)$ is $\sigma$--continuous, it is $s$-continuous by Lemma \ref{lem:bdd_sigma_cts_iff_bdd_s_cts_Sak_1_8_10}. The seminorm
  $x \mapsto \psi(px^{*}xp)^{1/2}$ is a defining seminorm for the $s$--topology on $M$. It follows that $x\mapsto \varphi(x(p_0+p))$ is $s$-continuous
  and therefore $\sigma$--continuous. This contradicts the maximality of $p_0$, and therefore $p_0=1$ and the result follows. 
\end{proof}