\section{WStarAlgebras and their Topologies}

In this section we collect basic definitions and results about $W^{*}$--algebras. 

An associative algebra $\mathcal{A}$ over the complex numbers is called a \textbf{normed algebra}
  if there is a map assigning for every $x\in \mathcal{A}$ a real number $\|x\|$, and this map satisfies:
  \begin{enumerate}
  \item $\|x\|\ge 0$ and $\|x\|=0 \iff x=0$;
  \item $\|x+y\|\le \|x\|+\|y\|$;
  \item $\|\lambda x\|= |\lambda|\|x\|$;
  \item $\|xy\|\le \|x\|\|y\|$.
  \end{enumerate} 
  If $\mathcal{A}$ is also complete with respect to the norm, it is called a \textbf{Banach algebra}. 
  A mapping $*: \mathcal{A}\to \mathcal{A}$ is called an \textbf{involution} if it satisfies:
  \begin{enumerate}
  \item $(x^{*})^{*}=x$;
  \item $(x+y)^{*}=x^{*}+y^{*}$;
  \item $(xy)^{*}=y^{*}x^{*}$;
  \item $(\lambda x)^{*}=\overline{\lambda}x^{*}$.
  \end{enumerate}
  where $\lambda$ above is a complex number.

\begin{definition}
  \label{def:cstar_def}
  A Banach algebra $\mathcal{A}$ with an involution is a $\mathbf{C^{*}}$\textbf{--algebra} if the relation 
  $\|x^{*}x\|=\|x\|^{2}$ holds for every $x\in \mathcal{A}$.
\end{definition}

\begin{theorem}
  \label{thm:cstar_unital_iff_extreme_Sak_1_6_1} (Sakai 1.6.1)
  A $C^{*}$--algebra is unital if and only if its closed unit ball has an extreme point.
\end{theorem}

\begin{proof}
\end{proof}

\begin{definition}
  \label{def:wstar_def}
  \uses{def:cstar_def}
  A $C^{*}$--algebra $M$ is called a $\mathbf{W^{*}}$\textbf{--algebra} if it is a dual Banach space, i.e. if there exists a Banach space
  $M_{*}$ so that $(M_{*})^{*}$ is isometrically isomorphic to $M$. We call $M_{*}$ a \textbf{predual} of $M$. 
\end{definition}

Note: It is premature to fix a notation like $M_{*}$ in the above, but the aim of this project is to prove that any two 
preduals of $M$ are isomorphic as Banach spaces, so we take the liberty with Sakai to ``fix a reference predual'' at the outset.

\begin{definition}
  \label{def:sigma_top}
  \uses{def:wstar_def}
  The weak-$*$ topology $\sigma(M,M_{*})$ on a $W^{*}$--algebra $M$ is called the $\sigma$--topology.
\end{definition}

\begin{lemma}
    \label{lem:sigma_closed_star_subalg_wstar} (Sakai 1.1.4)
    \uses{def:wstar_def,def:sigma_top}
  A self-adjoint, $\sigma$--closed $*$--subalgebra $N$ of a $W^{*}$--algebra $M$ is also a $W^{*}$--algebra, since $(M_{*}/N^{0})^{*}=N$, where
  $N^{0}$ is the polar (annihilator) of $N$ in $M_{*}$, i.e. the set of $\varphi\in M_{*}$ such that $\varphi(x)=0$ for all $x\in N$.
\end{lemma}

\begin{proof}
\end{proof}

\begin{lemma}
  \label{lem:wstar_unital}
  \uses{thm:cstar_unital_iff_extreme_Sak_1_6_1}
  Every $W^{*}$--algebra is unital.
\end{lemma}

\begin{proof}
  The norm unit ball of $M=(M_{*})^{*}$ is closed in the $\sigma$--topology, is convex, and so has an extreme point by the Krein-Milman Theorem.
  By Theorem \ref{thm:cstar_unital_iff_extreme_Sak_1_6_1}, the result follows.
\end{proof}

\begin{lemma}
  \label{lem:cut_down_sigma_closed_cts} (Sakai 1.7.6)
  \uses{def:sigma_top}
  Let $p$ be any projection in $M$. Then the subalgebra $pMp$ is $\sigma$--closed and the mapping $x\mapsto pxp$ is $\sigma$--continuous.
\end{lemma}

\begin{proof}
\end{proof}

\begin{lemma}
  \label{lem:cut_down_of_wstar} 
  \uses{def:sigma_top}
  \uses{lem:cut_down_sigma_closed_cts,lem:sigma_closed_star_subalg_wstar}
  If $p\in M$ is a projection, $pMp$ is also a $W^{*}$--algebra with identity $p$.
\end{lemma}

\begin{proof}
  It is clear that $eMe$ is an algebra, since any sum or product of elements preserves the pre- and post- multiplication by $e$.
  It is also clear that $e$ is the multiplicative identity of this algebra. The product reversing property of $*$ and the fact that
  $e$ is a self-adjoint projection implies that $eMe$ is a $*$--algebra. By Lemma \ref{lem:sigma_closed_star_subalg_wstar}, if this 
  $*$--subalgebra is $\sigma$--closed, then it is a $W^{*}$--subalgebra of $M$. But this is the result of \ref{lem:cut_down_sigma_closed_cts}.
\end{proof}

\begin{lemma}
  \label{lem:left_mul_proj_right_mul_proj_sig_cts_Sak_1_7_7} (Sak 1.7.7)
  \uses{def:sigma_top}
  If $p$ is any projection of $M$, then the mappings $x\mapsto px$ and
  $x\mapsto xp$ are $\sigma$--continuous.
\end{lemma}

\begin{proof}
\end{proof}

\begin{lemma}
  \label{lem:star_left_mul_right_mul_sig_cts_Sak_1_7_8} (Sak 1.7.8)
  \uses{def:sigma_top}
  The mappings $x\mapsto x^{*},ax$, and $xa$ are $\sigma$--continuous for
  $x,a \in M$.
\end{lemma}

\begin{proof}
\end{proof}

\begin{corollary}
  \label{cor:sig_clos_of_cstar_is_wstar_Sak_1_7_9} (Sak 1.7.9)
  \uses{lem:left_mul_proj_right_mul_proj_sig_cts_Sak_1_7_7,lem:star_left_mul_right_mul_sig_cts_Sak_1_7_8,def:sigma_top}
  Let $H$ be a subset of $M$ and $\mathcal{A}$ be the $C^{*}$--subalgebra of $M$ generated
  by $H$, and let $\overline{\mathcal{A}}$ be the $\sigma(M,M_{*})$--closure of $\mathcal{A}$. Then
  $\overline{\mathcal{A}}$ is a $W^{*}$--subalgebra of $M$. $\overline{\mathcal{A}}$ is called a $W^{*}$--subalgebra
  of $M$ generated by $H$. Moreover, $\overline{\mathcal{A}}$ is commutative if $\mathcal{A}$ is commutative.
\end{corollary}

\begin{lemma}
  \label{lem:wstar_of_masa}
  \uses{cor:sig_clos_of_cstar_is_wstar_Sak_1_7_9}
  A maximal abelian $*$--subalgebra $\mathcal{A}$ of a $W^{*}$--algebra $M$ is also a $W^{*}$--algebra.
\end{lemma}

\begin{proof}
    By maximality, $\mathcal{A} =\overline{\mathcal{A}}$, and so Corollary \ref{cor:sig_clos_of_cstar_is_wstar_Sak_1_7_9} 
    implies $\mathcal{A}$ is a $W^{*}$--subalgebra of $M$.
\end{proof}