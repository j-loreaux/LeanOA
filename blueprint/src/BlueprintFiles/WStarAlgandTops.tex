\chapter{WStarAlgebras and their Topologies}

\section{Basic Definitions and Results}

In this section we collect basic definitions and results about $C^{*}$--algebras and $W^{*}$--algebras. 

An associative algebra $\mathcal{A}$ over the complex numbers is called a \textbf{normed algebra}
  if there is a map assigning for every $x\in \mathcal{A}$ a real number $\|x\|$, and this map satisfies:
  \begin{enumerate}
  \item $\|x\|\ge 0$ and $\|x\|=0 \iff x=0$;
  \item $\|x+y\|\le \|x\|+\|y\|$;
  \item $\|\lambda x\|= |\lambda|\|x\|$;
  \item $\|xy\|\le \|x\|\|y\|$.
  \end{enumerate} 
  If $\mathcal{A}$ is also complete with respect to the norm, it is called a \textbf{Banach algebra}. 
  A mapping $*: \mathcal{A}\to \mathcal{A}$ is called an \textbf{involution} if it satisfies:
  \begin{enumerate}
  \item $(x^{*})^{*}=x$;
  \item $(x+y)^{*}=x^{*}+y^{*}$;
  \item $(xy)^{*}=y^{*}x^{*}$;
  \item $(\lambda x)^{*}=\overline{\lambda}x^{*}$.
  \end{enumerate}
  where $\lambda$ above is a complex number.

\begin{definition}
  \label{def:cstar_def}
  A Banach algebra $\mathcal{A}$ with an involution is a $\mathbf{C^{*}}$\textbf{--algebra} if the relation 
  $\|x^{*}x\|=\|x\|^{2}$ holds for every $x\in \mathcal{A}$. A $C^{*}$--algebra is \textbf{unital} if it has a multiplicative
  identity.
\end{definition}

\begin{lemma}
  \label{lem:star_pres_norm_Sak_1_1_6} (Sak 1.1.6)
  \uses{def:cstar_def}
  The involution on a $C^{*}$--algebra preserves the norm.
\end{lemma}

\begin{proof}
  Note $\|x\|^{2}=\|x^{*}x\|\le \|x^{*}\|\|x\|$, and $\|x^{*}\|^{2}\le \|xx^{*}\|\le \|x\|\|x^{*}\|$.
\end{proof}

\begin{theorem}
  \label{thm:comm_cstar_cts_fcns_Sak_1_2_1} (Sak 1.2.1)
  \uses{def:cstar_def}
  Every commutative, unital $C^{*}$--algebra $\mathcal{A}$ is $*$--isomorphic to $C(K)$,
  the algebra of complex-valued continuous functions on $K$, where $K$ is the compact Hausdorff
  space of maximal ideals of $\mathcal{A}$. (Note $C(K)$ is itself a $C^{*}$--algebra with pointwise 
  complex conjugation as $*$ and the uniform norm.)
\end{theorem} 

\begin{proof}
  By Banach algebra theory, the Gelfand transform provides a homomorphism of $\mathcal{A}$ into $C(K)$.
  Standard calculation using $C^{*}$--identity gets that it is isometric. Mapping a self-adjoint element $h$
  to the unit circle (complex exponentiation) one can show that the spectrum lands on the unit circle and so 
  the spectrum of $h$ is in the reals. The spectrum of an element in $C(K)$ is its range, from which $*$--
  preservation follows. (I should have just written this proof out...will do later...)
\end{proof}

\begin{corollary}
  \label{cor:nonunital_gelfand_Sak_1_2_2} (Sak 1.2.2)
  \uses{thm:comm_cstar_cts_fcns_Sak_1_2_1}
  Let $\mathcal{A}$ be a commutative $C^{*}$--algebra without identity. Then $\mathcal{A}$ is $*$--isomorphic to the 
  algebra $C_{0}(\Omega)$ of continuous functions vanishing at infinity on a locally compact Hausdorff space $\Omega$.
  The space $\Omega$ is called the \textbf{spectrum space} of $\mathcal{A}$.
\end{corollary}

\begin{corollary}
  \label{cor:spec_sa_homeo_spec_cstar}
  \uses{cor:nonunital_gelfand_Sak_1_2_2}
  Let $\mathcal{A}$ be a commutative $C^{*}$--algebra generated by a single self-adjoint element $a$ and let $\Omega$ be
  the spectrum space of $\mathcal{A}$. Then $\Omega \cup \{\infty\}$ is homeomorphic to $\sigma(a)\cup \{0\}$ and 
  $\mathcal{A}=C_{0}(\sigma(a)\cup \{0\})$, the algebra of complex-valued continuous functions on $a$'s spectrum that vanish at $0$. 
\end{corollary}

\begin{proof}
  The function $\xi$ sending a point $t\in\Omega$ to $a(t)\in \sigma(a)$ and $\infty$ to $0$, where ``$a(\cdot)$'' is the image of 
  $a$ under the Gelfand transform. Since $\mathcal{A}$ is generated by $a$, the map is bijective. The map is continuous, since $a$ is a continuous
  function. We have a continuous bijection from a compact space (one-point compactification of $\Omega$) into a Hausdorff space $\sigma(a)\cup \{0\}$,
  and by Lemma \ref{lem:homeo_of_cts_bij_cpt_haus} this is a homeomorphism. (Need to clarify a few details here...)
\end{proof}

\begin{lemma}
  \label{lem:one_is_extreme}
  \uses{thm:comm_cstar_cts_fcns_Sak_1_2_1,lem:star_pres_norm_Sak_1_1_6}
  In a unital $C^{*}$--algebra $\mathcal{A}$, the identity $1$ is an extreme point of the closed unit sphere $S$.
\end{lemma}

\begin{proof}
  Suppose $a,b\in S$ and $1=\frac{a+b}{2}$. Then, since $1=1^{*}$, we have $1=\frac{a^{*}+b^{*}}{2}$ and one checks that $1=\frac{c+d}{2}$,
  where $c=\frac{a+a^{*}}{2}$ and $d=\frac{b+b^{*}}{2}$. Now, $d=2-c$ and so $d$ and $c$ commute and each are self-adjoint (hence normal) 
  and therefore $C^{*}\{1,d,c\}$ is commutative and can be identified with $C(K)$ for a compact Hausdorff space $K$. 
  The self-adjointness of $c$ and $d$, the triangle inequality and Lemma \ref{lem:star_pres_norm_Sak_1_1_6} imply 
  that the range of the associated functions $f,g$ are in $[-1,1]$. In order for $f(t)+g(t)=2$ for all $t\in K$ 
  it must be that $f(t)$ and $g(t)$ are always $1$, and thus $c=d=1$. Now $2=2c=a+a^{*}$ and so $a=2-a^{*}$ and so $a$ is normal and 
  so the unital $C^{*}$--subalgebra generated by $a$ is commutative.
  Again represent the unital $C^{*}$--algebra as $C(K)$ for some compact Hausdorff $K$. Let $f$ be the
  function corresponding to $a$. We have $\Re(f)(t)=1$ for all $t$, and hence $\Re(f)(t)^2=|f(t)|=1$ for all $t$ and $\Im(f)(t)=0$ for
  all $t$ and therefore $a=1$.
\end{proof}

\begin{lemma}
  \label{lem:fake_approx_unit}
  \uses{cor:nonunital_gelfand_Sak_1_2_2}
  Let $x$ be an extreme point of the norm unit ball of a $C^{*}$--algebra $\mathcal{A}$, so that the $C^{*}$--subalgebra generated by $x^{*}x$ is identified with 
  $C_{0}(\Omega)$ as in Lemma \ref{cor:nonunital_gelfand_Sak_1_2_2}. There exists a sequence $(y_n)$ of positive elements in $C_{0}(\Omega)$
  with $\|y_{n}\|\le 1$ for all $n$, such that $\|(x^{*}x)y_n-x^{*}x\|\to 0$ and $\|(x^{*}x)y_n^{2}-x^{*}x\|\to 0$ as $n\to \infty$.
\end{lemma}

\begin{proof}
\end{proof}

\begin{lemma}
  \label{lem:extreme_partial_isom}
  \uses{cor:nonunital_gelfand_Sak_1_2_2,lem:fake_approx_unit}
  If $x$ is an extreme point of the norm unit ball of a $C^{*}$--algebra $\mathcal{A}$, then $x^{*}x$ is a projection. I.e. $x$ is a partial
  isometry. 
\end{lemma}

\begin{proof}
\end{proof}


\begin{theorem}
  \label{thm:cstar_unital_iff_extreme_Sak_1_6_1} (Sakai 1.6.1)
  \uses{lem:one_is_extreme,lem:extreme_partial_isom}
  A $C^{*}$--algebra is unital if and only if its closed unit ball has an extreme point.
\end{theorem}

\begin{proof}
\end{proof}

\begin{proposition}
  \label{prop:extr_pos_projs_Sak_1_6_2} (Sakai 1.6.2)
  \uses{cor:nonunital_gelfand_Sak_1_2_2}
  Let $P$ be the set of positive elements in a $C^{*}$--algebra $\mathcal{A}$ with unit sphere $S$. Then the extreme
  points of $P\cap S$ are the projections of $\mathcal{A}$.
\end{proposition}

\begin{proof}
\end{proof}

\begin{proposition}
  \label{prop:extr_sa_sa_unitaries_Sak_1_6_3} (Sakai 1.6.3)
  \uses{prop:extr_pos_projs_Sak_1_6_2}
  Let $\mathcal{A}^{s}$ be the set of self-adjoint elements in a $C^{*}$--algebra $\mathcal{A}$ with unit sphere $S$. Then
  the extreme points of $\mathcal{A}^{s}\cap S$ is the set of all self-adjoint unitary elements of $\mathcal{A}$.
\end{proposition}

\begin{proof}
\end{proof}

\begin{definition}
  \label{def:wstar_def}
  \uses{def:cstar_def}
  A $C^{*}$--algebra $M$ is called a $\mathbf{W^{*}}$\textbf{--algebra} if it is a dual Banach space, i.e. if there exists a Banach space
  $M_{*}$ so that $(M_{*})^{*}$ is isometrically isomorphic to $M$. We call $M_{*}$ a \textbf{predual} of $M$. 
\end{definition}

Note: It is premature to fix a notation like $M_{*}$ in the above, but the aim of this project is to prove that any two 
preduals of $M$ are isomorphic as Banach spaces, so we take the liberty with Sakai to ``fix a reference predual'' at the outset.

\begin{lemma}
    \label{lem:sigma_closed_star_subalg_wstar} (Sakai 1.1.4)
    \uses{def:wstar_def,def:sigma_top}
  A self-adjoint, $\sigma$--closed $*$--subalgebra $N$ of a $W^{*}$--algebra $M$ is also a $W^{*}$--algebra, since $(M_{*}/N^{0})^{*}=N$, where
  $N^{0}$ is the polar (annihilator) of $N$ in $M_{*}$, i.e. the set of $\varphi\in M_{*}$ such that $\varphi(x)=0$ for all $x\in N$.
\end{lemma}

\begin{proof}
\end{proof}

\section{The Ultraweak (Sigma) Topology}

\begin{definition}
  \label{def:sigma_top}
  \uses{def:wstar_def}
  The weak-$*$ topology $\sigma(M,M_{*})$ on a $W^{*}$--algebra $M$ is called the $\sigma$--topology.
\end{definition}

\begin{lemma}
  \label{lem:wstar_unital}
  \uses{thm:cstar_unital_iff_extreme_Sak_1_6_1}
  Every $W^{*}$--algebra is unital.
\end{lemma}

\begin{proof}
  The norm unit ball of $M=(M_{*})^{*}$ is closed in the $\sigma$--topology, is convex, and so has an extreme point by the Krein-Milman Theorem.
  By Theorem \ref{thm:cstar_unital_iff_extreme_Sak_1_6_1}, the result follows.
\end{proof}

\begin{lemma}
  \label{lem:cut_down_sigma_closed_cts} (Sakai 1.7.6)
  \uses{def:sigma_top}
  Let $p$ be any projection in $M$. Then the subalgebra $pMp$ is $\sigma$--closed and the mapping $x\mapsto pxp$ is $\sigma$--continuous.
\end{lemma}

\begin{proof}
\end{proof}

\begin{lemma}
  \label{lem:cut_down_of_wstar} 
  \uses{def:sigma_top}
  \uses{lem:cut_down_sigma_closed_cts,lem:sigma_closed_star_subalg_wstar}
  If $p\in M$ is a projection, $pMp$ is also a $W^{*}$--algebra with identity $p$.
\end{lemma}

\begin{proof}
  It is clear that $eMe$ is an algebra, since any sum or product of elements preserves the pre- and post- multiplication by $e$.
  It is also clear that $e$ is the multiplicative identity of this algebra. The product reversing property of $*$ and the fact that
  $e$ is a self-adjoint projection implies that $eMe$ is a $*$--algebra. By Lemma \ref{lem:sigma_closed_star_subalg_wstar}, if this 
  $*$--subalgebra is $\sigma$--closed, then it is a $W^{*}$--subalgebra of $M$. But this is the result of \ref{lem:cut_down_sigma_closed_cts}.
\end{proof}

\begin{lemma}
  \label{lem:left_mul_proj_right_mul_proj_sig_cts_Sak_1_7_7} (Sak 1.7.7)
  \uses{def:sigma_top}
  If $p$ is any projection of $M$, then the mappings $x\mapsto px$ and
  $x\mapsto xp$ are $\sigma$--continuous.
\end{lemma}

\begin{proof}
\end{proof}

\begin{lemma}
  \label{lem:star_left_mul_right_mul_sig_cts_Sak_1_7_8} (Sak 1.7.8)
  \uses{def:sigma_top}
  The mappings $x\mapsto x^{*},ax$, and $xa$ are $\sigma$--continuous for
  $x,a \in M$.
\end{lemma}

\begin{proof}
\end{proof}

\begin{corollary}
  \label{cor:sig_clos_of_cstar_is_wstar_Sak_1_7_9} (Sak 1.7.9)
  \uses{lem:left_mul_proj_right_mul_proj_sig_cts_Sak_1_7_7,lem:star_left_mul_right_mul_sig_cts_Sak_1_7_8,def:sigma_top}
  Let $H$ be a subset of $M$ and $\mathcal{A}$ be the $C^{*}$--subalgebra of $M$ generated
  by $H$, and let $\overline{\mathcal{A}}$ be the $\sigma(M,M_{*})$--closure of $\mathcal{A}$. Then
  $\overline{\mathcal{A}}$ is a $W^{*}$--subalgebra of $M$. $\overline{\mathcal{A}}$ is called a $W^{*}$--subalgebra
  of $M$ generated by $H$. Moreover, $\overline{\mathcal{A}}$ is commutative if $\mathcal{A}$ is commutative.
\end{corollary}

\begin{lemma}
  \label{lem:wstar_of_masa}
  \uses{cor:sig_clos_of_cstar_is_wstar_Sak_1_7_9}
  A maximal abelian $*$--subalgebra $\mathcal{A}$ of a $W^{*}$--algebra $M$ is also a $W^{*}$--algebra.
\end{lemma}

\begin{proof}
    By maximality, $\mathcal{A} =\overline{\mathcal{A}}$, and so Corollary \ref{cor:sig_clos_of_cstar_is_wstar_Sak_1_7_9} 
    implies $\mathcal{A}$ is a $W^{*}$--subalgebra of $M$.
\end{proof}

\section{Other Topologies on WStarAlgebras}
 
In this section, let $M$ be a $W^{*}$--algebra.

\begin{proposition}
  \label{prop:loc_cvx_result}
  The set of $\sigma(M,M_{*})$--continuous linear functionals is precisely $M_{*}$.
\end{proposition}

\begin{proof}
  Standard result in lctvs.
\end{proof}

\begin{definition}
  \label{def:Mackey_top}
  The locally convex topology on $M$ generated by the seminorms $p_{K}(x)=\sup_{\varphi\in K}|\varphi(x)|$, ranging over 
  $K$ the $\sigma(M_{*},M)$--compact subsets of $M_{*}$, is called the \textbf{Mackey topology} on $M$.
  We denote this topology by $\tau(M,M_{*})$.
\end{definition}

The Mackey topology is the strongest topology on $M$ for which the functionals in $M_{*}$ are all continuous.

\begin{definition}
  \label{def:strong_top}
  The locally convex topology on $M$ generated by the seminorms $x\mapsto \varphi(x^{*}x)^{1/2}$ ranging over $\varphi\in M_{*}$ is 
  called the \textbf{strong topology}, or $s$--topology, on $M$. We denote it by $s(M,M_{*})$.
\end{definition}

\textbf{The following result additionally uses the Mackey-Arens Theorem}, so we may need to prove this as well. 
By $A\le B$ on topologies below, we mean $A$ is weaker than $B$.

\begin{theorem}
  \label{thm:sigma_le_s_le_tau}
  \uses{def:Mackey_top,def:strong_top,def:sigma_top}
  We have $\sigma(M,M_{*})\le s(M,M_{*})\le \tau(M,M_{*})$.
\end{theorem}

\begin{proof}
\end{proof}

\begin{corollary}
  \label{cor:sigma_eq_s_on_ball}
  \uses{thm:sigma_le_s_le_tau}
  The $\sigma$--topology is equivalent to the $s$--topology on the unit ball of $M$, and hence on bounded sets.
\end{corollary}

\begin{proof}
\end{proof}

