\section{Stonean Spaces and Maximal Commutative Subalgebras}

Recall that a compact Hausdorff space is \textbf{Stonean} if the closure of every open set is open. We note that Lemma
\ref{lem:fin_lin_approx_of_stonean_Sak_1_3_1} is only part of Sakai 1.3.1, since that is all
we need for the proof of Theorem \ref{lem:sigma_cont_of_normal_Sak_1_7_4}. If $K$ is a topological space in which singletons are closed,
and $p$ is a projection in $C(K)$, the spectrup of $p$ is its range and is contained in $\{0,1\}$.
Furthermore, $p^{-1}\{0\}=K\backslash p^{-1}\{1\}$, and so both of these sets are clopen, and identify connected components of $K$.
If $K$ is Stonean, there are many such connected components:

\begin{lemma}
  \label{lem:fin_lin_approx_of_stonean_Sak_1_3_1} (Sak  1.3.1)
  Let $K$ be a Stonean space. Then every positive self-adjoint element $a$ in $C(K)$ can be uniformly approximated by 
  finite linear combinations of projections in $C(K)$ having nonnegative coefficients. 
\end{lemma}

\begin{proof}
Since $a$ is self-adjoint, its spectrum is its range in $\mathbb{R}_{\ge 0}$, and is therefore contained in $[0, \|a\|]$,
where $\|a\|=\sup_{t\in K}|a(t)|$. Let $\varepsilon >0$ and choose real numbers
$0<\lambda_1<\lambda_2<\ldots<\lambda_n < \|a\|+1$ so that $|\lambda_{i+1}-\lambda_{i}|<\varepsilon$.
Now let $G_1= \overline{\{t\in K | a(t)<\lambda_1\}}$ and $G_i=\overline{\{t\in K
\backslash\bigcup_{j=1}^{i-1}G_{i}|a(t)<\lambda_i\}}$ for $i>1$. These are pairwise disjoint,
and $\chi_{G_{i}}$ are each continuous projections. Furthermore, for all $t\in K$,
\begin{align}
    |a(t)-\sum_{i=1}^{n}\lambda_i\chi_{G_{i}}(t)|<\varepsilon
\end{align}
and therefore $\|a-\sum_{i=1}^{n}\lambda_i\chi_{G_{i}}\|<\varepsilon$.
\end{proof}

In Sakai, the pointwise order on $C(K)$ in Proposition \ref{prop:stonean_of_cts_fcns_incr_cond_complete_Sak_1_3_2}
isn't specified beforehand in the book. Following $C^{*}$--algebra theory, $a\ge 0$ in $C(K)$
iff the range of $a$ is contained in $\mathbb{R}_{\ge 0}$, and $b\le a$ iff $b-a \ge 0$. The spectrum 
of a function in $C(K)$ is just its range. So $b\le a$ iff $b(t)\le a(t)$ for all $t\in K$. A net
 $(f_{\alpha})$ in $C(K)$ is \textbf{bounded} if there is a positive constant that bounds the norms of all $f_{\alpha}$.
A net is said to be \textbf{increasing} if $\alpha \le \beta$ implies $f_{\alpha}\le f_{\beta}$ for all $\alpha,\beta$
in the associated directed set. We ought to note, here, that Sakai opts not to use net language
(although he sort of does when introducing subscripts) since his directed set is identical to his net
in Proposition \ref{prop:stonean_of_cts_fcns_incr_cond_complete_Sak_1_3_2}. Recall that 
the \textbf{support} of $f\in C(K)$ is $\overline{\{x\in K | f(x)\ne 0\}}$.

\begin{lemma}
    \label{lem:cts_fns_ge_zero_le_one_directed}
    Let $K$ be a compact Hausdorff space, with $U\subseteq K$ open, then the set of $f\in C(K)$ supported on $U$ such that
    $0 \le f \le 1$ is a directed set w.r.t. the order on $C(K)$.
\end{lemma}

\begin{proof}
    The usual order is a partial order, and given $f,g\in C(K)$ supported on $U$ we have $f\vee g \in C(K)$, 
    supported on $U$, where $f\vee g(t)=\max\{f(t),g(t)\}$.
\end{proof}

\begin{proposition}
    \label{prop:stonean_of_cts_fcns_incr_cond_complete_Sak_1_3_2} (Sakai 1.3.2)
    \uses{lem:cts_fns_ge_zero_le_one_directed,lem:selfadjoint_le_norm}
    Let $K$ be a compact Hausdorff space. Suppose every bounded increasing net
    of real valued, non-negative functions in $C(K)$ has a least upper bound in $C(K)$.
    Then $K$ is Stonean. 
\end{proposition}

\begin{proof}
    Suppose for the sake of a contradiction that $K$ isn't Stonean, i.e. that there exists an
    open set $U$ whose closure isn't open, i.e. that $\overline{U}^{c}$ isn't closed.
    Consider the set of $f$ in $C(K)$ with support contained in $U$ and such that $0\le f\le 1$. 
    By Lemma \ref{lem:selfadjoint_le_norm} and Lemma \ref{lem:cts_fns_ge_zero_le_one_directed}, 
    this is a bounded, increasing net with respect to the order on $C(K)$. 
    For any point $t\in U$, by Urysohn's Lemma (compact Hausdorff Spaces are regular) there
    exists a continuous function $g\in C(K)$ such that $g(t)=1$ and $g$ is zero on $U^{c}$. 
    It follows that if $f_s$ is the least upper bound of the above net, then $1=g(t)\le f_s(t)$, 
    and therefore (since we can play this game for any $t\in U$) it must be that $f_s(t)=1$ for any $t\in U$. 
    By continuity, $f_s(t)=1$ for all $t\in\overline{U}$. Now, since $\overline{U}^{c}\subset U^{c}$, 
    $f_c$ is identically zero on $U^{c}$. Since $\overline{U}^{c}$ is not closed, it must have a limit point
    $p\in \overline{U}$, and by continuity $f_c(p)=0$ But $f_c(p)=1$ by the above, which is a contradiction.
\end{proof}

\begin{lemma}
  \label{lem:spec_masa_wstar_stonean_Sak_1_7_5} (Sak 1.7.5)
  \uses{prop:stonean_of_cts_fcns_incr_cond_complete_Sak_1_3_2}
  If $C$ is any maximal commutative $C^*$--subalgebra of the $W^{*}$--algebra $M$, its
  spectrum space (maximal ideal space) is Stonean.
\end{lemma}

\begin{proof}
  Suppose $(a_{\lambda})$ is a uniformly bounded, increasing net of positive elements in $C$ with supremum $a$. 
  Given any unitary $u\in C$, we have $u^{*}\sup_{\lambda}a_{\lambda} u=\sup_{\lambda}u^{*}a_{\lambda}u$ and since every
  element in $C$ is a linear combination of unitary elements from $C$, we have that $\sup_{\lambda}a_{\lambda}\in C$. The result
  follows from Proposition \ref{prop:stonean_of_cts_fcns_incr_cond_complete_Sak_1_3_2} after applying the Gelfand transform.
\end{proof}
