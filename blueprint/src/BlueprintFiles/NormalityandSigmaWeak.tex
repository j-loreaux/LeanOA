\section{Normality and Ultraweak Continuity for Positive Functionals}

In what follows, let $M$ be a (nonzero) $W^{*}$--algebra. Let $\mathcal{P}(M)$
denote the projection lattice of $M$. Let $\varphi$ be a positive linear functional $\varphi$ on $M$.
We say $\varphi$ is \textbf{normal} if whenever $(p_{\alpha})$ is an increasing net of projections in $M$ with supremum $p$,
we have $\varphi(p_{\alpha})\to \varphi(p)$. In this section we show that this property is equivalent to 
$\sigma(M, M_{*})$--continuity. We say that a linear functional $\varphi$ on $M$
is \textbf{positive} if $\varphi(x)\ge 0$ whenever $x\ge 0$.

\subsection{Ultraweakly Continuous Implies Normal} 

In what follows, let $T$ denote the set of all $\sigma$--continuous positive
linear functionals on $M$, and $E$ the linear space of all finite linear combinations
of elements of $T$. Let $P$ denote the set of positive elements of $M$.

\begin{lemma}
  \label{lem:homeo_of_cts_bij_cpt_haus}
  \lean{isHomeomorph_iff_continuous_bijective}
  \mathlibok
  Every continuous bijection from a compact space to a Hausdorff space is a 
  homeomorphism.
\end{lemma}

\begin{proof}
  \leanok
  Omitted. In Mathlib.
\end{proof}

\begin{theorem}
  \label{thm:Banach_Alaoglu}
  \lean{WeakDual.isCompact_closedBall}
  \mathlibok
  (Banach-Alaoglu) The closed unit ball of the dual space of a normed vector space is compact in the
  weak-$*$ topology.
\end{theorem}

\begin{proof}
  \leanok
  Omitted. In Mathlib.
\end{proof}

\begin{theorem}
  \label{lem:sigma_cont_of_normal_Sak_1_7_4} (Sak 1.7.4)
  \uses{thm:Banach_Alaoglu,lem:homeo_of_cts_bij_cpt_haus,lem:uw_pos_sep_pts,lem:selfadjoint_le_norm,def:sigma_top}
  Every uniformly bounded, increasing net in $M^{s}$ converges
  to its least upper bound in the $\sigma$--topology. Further, if
  $x=\sup_{\lambda}x_{\lambda}$ then $a^{*}xa=\sup_{\lambda}a^{*}x_{\lambda}a$.
\end{theorem}

\begin{proof}
  Letting $T$ denote the set of $\sigma$--continuous positive functionals on $M$ and $E$ its linear span, by
  Lemma \ref{lem:uw_pos_sep_pts}, the subspace $E$ separates the points of $M$ and so the $\sigma(M,E)$
  is Hausdorff. By the Banach-Alaoglu Theorem (\ref{thm:Banach_Alaoglu}) the closed unit ball $B$ of $M$ is
  compact in the $\sigma(M,M_{*})$--topology. Since $\sigma(M,E)$ is weaker than $\sigma(M,M_{*})$, the identity
  map on $B$ is $\sigma(M,M_{*})-\sigma(M,E)$--continuous, and therefore a homeomorphism, 
  by Lemma \ref{lem:homeo_of_cts_bij_cpt_haus}. So, to prove that a uniformly bounded net $(x_{t})$ is $\sigma$--
  Cauchy, it's enough to check that for all $\varepsilon>0$ and $\varphi\in T$ that there is a $\Lambda$ such that
  $\varphi(x_{\lambda}-x_{\mu})\le \varepsilon$ for $\lambda,\mu$ with $\lambda \ge \Lambda$ and $\mu \ge \Lambda$.
  
  If $(x_{\lambda})$ is a uniformly bounded increasing net in $M^{s}$, and $\varphi\in T$, then $(\varphi(x_{\lambda}))$ is
  a uniformly bounded increasing net of real numbers and therefore is Cauchy. By the above paragraph, $(x_{\lambda})$ is
  $\sigma$--Cauchy and by the $\sigma$--compactness of $B$ and the fact that $(x_{\lambda})$ is increasing there exists $x\in B$
  so that $x_{\lambda} \to x$ in the $\sigma$--topology. Since for any $\varphi\in T$ and any $\lambda$ we have 
  $0 \le \varphi(x-x_{\lambda})$, by Lemma \ref{lem:non_pos_elem_neg_for_some_state_Sak_1_7_2} we have $0\le x - x_{\lambda}$.
  Therefore $\sup(x_{\lambda})\le x$. For a given $\varphi\in T$ and $\varepsilon = \varphi(x-\sup(x_{\lambda}))$ one find 
  $x_{\lambda}$ so that $\varphi(x-x_{\lambda})<\varphi (x - \sup(x_{\lambda}))$, which is a contradiction, therefore $x=\sup(x_{\lambda})$.

  By Lemma \ref{lem:star_conj_pos} and the fact that $(u^{-1})^{*}=(u^{*})^{-1}$, then $u^{*}x_{\lambda}u\le u^{*}xu$
  is equivalent to $x_{\lambda}\le x$ and so $\sup(u^{*}x_{\lambda}u)=u^{*}\sup(x_{\lambda})u=u^{*}xu$ if $u\in M$ is invertible.

  If $a\in M$ is not invertible, one can shift the spectrum horizontally to avoid zero, i.e. there is a $c>0$ so that $c1+a$ is invertible. Indeed
  observe that $a-t1=(a+c1)-(t+c)1$ and therefore $t\in \sigma(a)\iff t+c \in \sigma(a+c1)$. In particular if we let $t=-c$ here we find that
  $0\in \sigma(a)+c \iff 0 \in \sigma(a+c1)$.
  By the invertible case above, 
  \begin{align}
    c^2\varphi(x_{\lambda})+c\varphi(a^{*}x_{\lambda}+x_{\lambda}a)+\varphi(a^{*}x_{\lambda}a)&=\varphi((c1+a)^{*}x_{\lambda}(c1+a))\\\nonumber
    &\to \varphi((c1+a)^{*}x(c1+a))
  \end{align}
  for all $\varphi\in T$. 
  It suffices to show that $\varphi(a^{*}x_{\lambda}+x_{\lambda}a)\to \varphi(a^{*}x+xa)$ for all $\varphi\in T$, 
  because then $\varphi(a^{*}x_{\lambda}a)\to \varphi(a^{*}xa)$ by the above. We can use Cauchy-Schwartz for this as follows. If $\alpha\le \beta$,
  we have $x_{\beta}-x_{\alpha}\ge 0$, and since the net $(x_{\lambda})$ is uniformly bounded there exists $M>0$ such that (using \ref{lem:selfadjoint_le_norm})
  \begin{align}
    |\varphi(a^{*}(x_{\beta}-x_{\alpha}))|&=|\varphi(a^{*}(x_{\beta}-x_{\alpha})^{1/2}(x_{\beta}-x_{\alpha})^{1/2})|\\\nonumber
                                          &\le \varphi(a^{*}(x_{\beta}-x_{\alpha})a)^{1/2}\varphi((x_{\beta}-x_{\alpha}))^{1/2}\\\nonumber
                                          &\le (2M\varphi(a^{*}a)^{1/2}\varphi((x_{\beta}-x_{\alpha}))^{1/2}
  \end{align}
  and by the positivity of $\varphi$ we have $|\varphi(a^{*}(x_{\beta}-x_{\alpha}))|=|\varphi((x_{\beta}-x_{\alpha})a)|$.
  It follows that $\varphi(a^{*}x_{\lambda}+x_{\lambda}a)\to \varphi(a^{*}x+xa)$, and the proof is finished.
\end{proof}

\subsection{Normal Implies Ultraweakly Continuous}

\begin{lemma}
  \label{lem:proj_of_sup_proj}
  \uses{lem:sigma_cont_of_normal_Sak_1_7_4}
  For every increasing bounded net $(p_{\alpha})$ of projections in $M$, the supremum $p$
  is a projection in $M$
\end{lemma}

\begin{proof}
\end{proof}

If $P:\mathcal{P}(M)\to \text{Prop}$ is a predicate, the usual ordering ``$\le$'' on projections
induces an order on the set $\{p\in \mathcal{P}(M)|P(p)\}$. In what follows there will be no confusion if 
we also denote this induced order by ``$\le$''.

\begin{lemma}
  \label{lem:exists_uw_ge_normal}
  \uses{lem:uw_pos_sep_pts}
  For every positive normal linear functional $\varphi$ and nonzero $p\in \mathcal{P}(M)$
  there exists a positive $\sigma(M,M_{*})$--continuous linear functional $\psi$ such that $\varphi(p)<\psi(p)$.
\end{lemma}

\begin{proof}
  By Lemma \ref{lem:uw_pos_sep_pts} there is a $\sigma(M,M_{*})$--continuous positive linear
  functional $\psi_0$ on $M$ such that $\psi_0(p)\ne 0$. Rescale this functional by a positive constant to obtain
  $\psi$ such that $\varphi(p)<\psi(p)$.
\end{proof}

\begin{lemma}
  \label{lem:msr_th_lemma}
  \uses{lem:sigma_cont_of_normal_Sak_1_7_4,lem:proj_of_sup_proj,cor:proj_sub_of_subproj}
For all positive linear functionals $\varphi,\psi$ with
$\varphi$ normal and $\psi$ $\sigma(M,M_{*})$--continuous and every nonzero $p\in \mathcal{P}(M)$
such that $\varphi(p)<\psi(p)$, there exists a nonzero $p_1 \in \mathcal{P}(M)$ such that $p_1\le p$ and for all nonzero
$q\in \mathcal{P}(M)$ with $q\le p_1$, we have $\varphi(q)<\psi(q)$. 
\end{lemma}

\begin{proof}
  We proceed by contradiction. Suppose the conclusion does not hold. Then, for every nonzero subprojection
  $p_1$ of $p$ there is a nonzero subprojection $q\le p_1$ such that $\varphi(q)\ge \psi(q)$. In particular,
  (letting $p_1=q$) there is a $q\le p$ such that $\varphi(q)\ge \psi(q)$. If $(q_\alpha)$ is a chain
  of such nonzero projections, $q_{\alpha}\to \sup q_{\alpha}$ in the $\sigma$--topology by Theorem \ref{lem:sigma_cont_of_normal_Sak_1_7_4}, 
  and by Lemma \ref{lem:proj_of_sup_proj} we know this supremum is a subprojection of $p_1$. Since $\varphi$
  is positive and normal, and $\psi$ is positive and $\sigma$--continuous, we have $\varphi(\sup q_{\alpha})\ge \psi(\sup q_{\alpha})$.
  Therefore, by Zorn's Lemma, there is a maximal $q_{0}\le p$ such that $\varphi(q_0)\ge \psi(q_0)$. We claim $q_{0}=p$. If not, 
  $p-q_0$ is a nonzero subprojection of $p$ by Corollary \ref{cor:proj_sub_of_subproj}, and there exists a nonzero projection $q_{1}\le p-q_{0}$
  such that $\varphi(q_{1})\ge \psi(q_{1})$. But then $q_{0}$ is a proper subprojection of $q_0+q_1$ and by linearity and positivity 
  $\varphi(q_0+q_1)\ge \psi(q_0+q_1)$ contradicting the maximality of $q_0$. Thus $p=q_0$ and so $\varphi(p) \ge \psi(p)$, contradicting 
  the hypothesis $\varphi(p)<\psi(p)$.  
\end{proof}

Note that the following likely contains the most crucial use of the normality of $\varphi$ for the later proof 
of Theorem \ref{thm:sigma_cts_of_normal_Sak_1_13_2}.

\begin{lemma}
  \label{lem:zorn_base}
  \uses{cor:proj_sub_of_subproj,lem:selfadjoint_le_norm,lem:sigma_cont_of_normal_Sak_1_7_4,lem:proj_of_sup_proj,def:sigma_top}
  Let $\varphi$ be a normal positive linear functional on $M$ and consider the predicate $P:\mathcal{P}(M)\to \text{Prop}$ defined, for $p\in \mathcal{P}(M)$, by 
  ``$M\ni x \mapsto \varphi(xp)$ is $\sigma(M,M_{*})$--continuous''. If $(p_{\alpha})$ is a chain of projections in $M$ such that $P(p_{\alpha})$ is true for each $\alpha$,
  then $P(\sup(p_{\alpha}))$ is true. Hence by Zorn's Lemma there is a maximal $p_0\in \mathcal{P}(M)$ such that $P(p_0)$ is true.  
\end{lemma}

\begin{proof}
  Let $x$ be on the unit sphere and let $p$ be the supremum of the $p_{\alpha}$. By Theorem \ref{lem:sigma_cont_of_normal_Sak_1_7_4} and Lemma \ref{lem:proj_of_sup_proj} 
  we know $p_{\alpha}$ converges in $\sigma$--topology to $p$, and $p$ is a projection. 
  By Corollary \ref{cor:proj_sub_of_subproj} we know $p-p_\alpha$ is a projection. By the $C^{*}$--property
  of the norm ($\|x^{*}x\|=\|x\|^{2}$), Lemma \ref{lem:selfadjoint_le_norm} and the positivity of $\varphi$ we have $\varphi(x^{*}x)\le \varphi(1)$.
  Now these facts together with Cauchy-Schwartz and the monotonicity of square roots, 
  \begin{align}
    |\varphi(x(p-p_{\alpha}))|&\le \varphi(x^{*}x)^{1/2}\varphi(p-p_{\alpha})^{1/2}\\ \nonumber
    &\le \varphi(1)^{1/2}\varphi(p-p_{\alpha})^{1/2}.
  \end{align}
  Hence, by the definition of operator norm, $\|\varphi(\cdot(p-p_{\alpha}))\|\le \varphi(1)^{1/2}\varphi(p-p_{\alpha})^{1/2}$. The right
  hand side converges to 0 with $\alpha$ due to the normality of $\varphi$ and therefore $\varphi(\cdot p_{\alpha})$ converges to $\varphi(\cdot p)$
  in norm. Since the set of $\sigma$--continuous functionals on $M$ is norm closed, it follows that $\sigma(\cdot p)$ is $\sigma$--continuous.
  We obtain a maximal $p_{0}$ by Zorn's Lemma.
\end{proof}

\begin{lemma}
  \label{lem:bdd_sigma_cts_iff_bdd_s_cts_Sak_1_8_10} (Sak 1.8.10)
  \uses{def:sigma_top}
  A linear functional $\rho$ on $M$ is $\sigma$--continuous on the unit sphere (hence $\sigma$--continuous) if and only if it is $s$--continuous on the
  unit sphere (hence $s$--continuous).
\end{lemma}

\begin{proof}
\end{proof}

\begin{theorem}
  \label{thm:sigma_cts_of_normal_Sak_1_13_2} (Sak 1.13.2)
  \uses{def:sigma_top}
  \uses{lem:zorn_base,lem:exists_uw_ge_normal,lem:msr_th_lemma,lem:cut_down_of_wstar,lem:spec_masa_wstar_stonean_Sak_1_7_5,lem:fin_lin_approx_of_stonean_Sak_1_3_1,lem:bdd_sigma_cts_iff_bdd_s_cts_Sak_1_8_10,lem:wstar_of_masa}
  Every positive normal linear functional $\varphi$ on $M$ is $\sigma(M,M_{*})$--continuous.
\end{theorem}

\begin{proof}
  The claim is obvious for the zero functional. Let $\varphi$ be a nonzero positive normal linear functional. By Lemma \ref{lem:zorn_base} we have
  a maximal $p_0\in \mathcal{P}(M)$ such that $M\ni x \mapsto \varphi(xp_0)$ is $\sigma(M,M_{*})$--continuous. Assume for the purposes of finding
  a contradiction that $p_0\ne 1$. By Lemma \ref{lem:exists_uw_ge_normal}
  there is a $\sigma(M,M_{*})$--continuous positive functional $\psi$ on $M$ such that $\varphi(1-p_0)<\psi(1-p_0)$. By Lemma \ref{lem:msr_th_lemma}
  there is a nonzero subprojection $p\le 1-p_0$ in $M$ such that $\varphi(q)< \psi(q)$ for every nonzero $q\le p$ in $M$. Let $x\in pMp$ be on the unit sphere. 
  Then $x^{*}x$ is positive and hence normal, so the $C^{*}$--subalgebra of $pMp$ generated by $x^{*}x$ and $p$ is commutative, and is hence contained in a maximal abelian
  $*$--subalgebra $A$ of $pMp$. Now $A$ is a $W^{*}$--subalgebra of $pMp$ by Lemma \ref{lem:wstar_of_masa} and hence is a maximal commutative $C^{*}$--subalgebra of $pMp$. Via the Gelfand Transform, 
  $A$ is star isomorphic to $C(K)$, where $K$ is Stonean by Lemmas \ref{lem:cut_down_of_wstar} and \ref{lem:spec_masa_wstar_stonean_Sak_1_7_5}. 
  By Lemma \ref{lem:fin_lin_approx_of_stonean_Sak_1_3_1} it follows that $\varphi(a)\le \psi(a)$ for every $a\ge 0$ in $A$, which holds a fortiori for $a\ge 0$ in $C^{*}(x^{*}x,p)$. 
  In particular, $\varphi(px^{*}xp)\le \psi(px^{*}xp)$. Therefore,
  \begin{align}
    |\varphi(x(p_0+p))|&\le |\varphi(xp_0)|+|\varphi(xp)|\\ \nonumber
    &\le |\varphi(xp_0)|+\varphi(1)^{1/2}\varphi(px^{*}xp)^{1/2}\\ \nonumber
    &\le |\varphi(xp_0)|+\varphi(1)^{1/2}\psi(px^{*}xp)^{1/2}.
  \end{align}
  Since $x\mapsto \varphi(xp_0)$ is $\sigma$--continuous, it is $s$-continuous by Lemma \ref{lem:bdd_sigma_cts_iff_bdd_s_cts_Sak_1_8_10}. The seminorm
  $x \mapsto \psi(px^{*}xp)^{1/2}$ is a defining seminorm for the $s$--topology on $M$. It follows that $x\mapsto \varphi(x(p_0+p))$ is $s$-continuous
  and therefore $\sigma$--continuous. This contradicts the maximality of $p_0$, and therefore $p_0=1$ and the result follows. 
\end{proof}